
%----------------------------------------------------------------------%
\section{Graphical Procedures}

Common \texttt{R} functions to create plots include \texttt{\textbf{plot()}}, \texttt{\textbf{boxplot()}}, \texttt{\textbf{barplot()}}, \texttt{\textbf{pairs()}}, \texttt{\textbf{matplot()}} , \texttt{\textbf{hist()}}, \texttt{\textbf{image()}}, \texttt{\textbf{contour()}}. Each of these plot functions are a part of the \textbf{base} package. To see what each command does, use the \texttt{help} file

\begin{framed}
\begin{verbatim}
help(matplot)
\end{verbatim}
\end{framed}

\noindent The \texttt{plot()} function is the most basic plotting function. Using the \texttt{plot()} function with only one variable will result in an index plot. 


 

\section{Scatterplots (with \texttt{locator} and \texttt{identify})}
\noindent Using two variables in the \texttt{plot()} function will result in scatter-plot, a graph with the first argument as the variable plotted on the x-axis and the second argument as the variable plotted on the y-axis.
\begin{framed}
\begin{verbatim}
X=c(4.53, 0.21, 3.42, 9.19, 6.17, 1.91, 7.9, 11.8, 1.06, 2.7)
Y=c(14.42, 2.52, 6.56, 15.6, 9.94, 9.82, 14.32, 10.71, 3.02, 13.82)

plot(X,Y,pch=18,col="red")
\end{verbatim}
\end{framed}


\begin{center}
\includegraphics[scale=0.35]{IDent1}
\end{center}
%---Insert Graphs (IDent1)

\subsection{The \texttt{locator()} function}
We can use the  \texttt{\textbf{locator()}} command to find out more about the position of observations. Suppose we are looking for the information on one only point, so we specify ``1" as an argument. (let's choose the first point from the right-handside in this instance).


\noindent Type the following, then click on the point, on the scatter-plot using your mouse (i.e. precisely on the point).


\begin{framed}
\begin{verbatim}
locator(1)
\end{verbatim}
\end{framed}
The XY co-ordinates of where the mouse was clicked will appear on screen.
\begin{framed}
\begin{verbatim}
> locator(1)
$x
[1] 11.76287

$y
[1] 10.7516
\end{verbatim}
\end{framed}

\subsection{The \texttt{identify()} function}
Suppose we wish to identify which case (ie. row number) that a data-point corresponds to.
For this we use the \textbf{\texttt{identify()}} command. We specify the X and Y variables as arguments (It is possible to have more than one pair of predictor and response variables on the same scatterplot).

\begin{framed}
\begin{verbatim}
identify(X,Y)
\end{verbatim}
\end{framed}

\noindent The observation corresponds to the 8th case. Find out which case corresponds to the lowest data point (i.e. one with the smallest ``y-value"). 
\newpage
%-------------------------------------------------- %
\section{Enhancing plots}

%\subsection{Vertical and Horizontal Range}
%\texttt{R} will automatically the range of the plot, as determined from the data.
%Should a different range be required for some reason, the horizontal range may be specified using the argument \texttt\textbf{{xlim=c(xmin, xmax)}} for some values $(xmin, xmax)$.
%Similarly the vertical range can be specified using the argument \texttt\textbf{{ylim=c(ymin, ymax)}}, again for some specified values $(ymin, ymax)$.


\subsection{Axes Limits}
The \textbf{\texttt{plot()}} function will automatically fit the size of the plot to the data. Should a larger plot area be required, the limits of the x-axis and y-axis can be specified either separately or in conjuction. 


\noindent Use the \textbf{\texttt{range()}} function to determine suitable limits. Using the arguments \textbf{\texttt{``xlim ="}} and \textbf{\texttt{``ylim ="}} you can adjust the limits, specifing the appropriate lower and upper limits as a vector: \textbf{\texttt{xlim=c(xmin, xmax)}} and \textbf{\texttt{ylim=c(ymin, ymax)}}

\begin{framed}
\begin{verbatim}
set.seed(1234)
x=rnorm(10)
y=rnorm(10)

plot(x,y) # plot limits fitted to data

range(x)
# -2.345698  1.084441

range(y)
# -0.9983864  2.4158352

plot(x,y,  xlim=c(-2.9,1.4),ylim=c(-1.5,3), cex=1.5,pch=18,col="red")



\end{verbatim}
\end{framed}
\begin{figure}[h!]
\centering
\includegraphics[width=0.5\linewidth]{./basicscatterplot}

\end{figure}



\subsection{Plot Characters}
The default plot character used in scatterplots and normal probability plots is a hollow white circle.
A different plot character may be specified by the \texttt{pch=} argument. The plot characters are specified by a number. The most useful are numbers 14 to 18.

\begin{figure}
\centering
\includegraphics[width=0.7\linewidth]{./symbols}
\caption{}
\label{fig:symbols}
\end{figure}
\subsection{Size of Plot Characters}
\begin{itemize}
\item Plot character size can be controlled by the \texttt{cex} parameters. The default size is 1. To double the size of a plot character, specify \texttt{cex=2}.

\item Plot character colour is controlled by the \texttt{col} argument. To have red plot characters, simply specify \texttt{col="red"}.
\end{itemize}

\subsection{Adding Points to a Scatterplot}
The \texttt{R} command \textbf{\texttt{points()}} is used to have different group of points, with different aesthetic qualities. This is particularly useful for depicting the respective distributions of subgroups.


Recall that the \textbf{\textit{iris}} data set can be subdivided into three species types: \textit{setosa}, \textit{virginica} and \textit{versicolor}.
It is convenient to plot the entire data set first, with basic aesthetic properties, then overwrite the points with the appropriate aesthetics for each subgroup. First we create three subsets of the \textbf{iris} data set.
\begin{framed}
\begin{verbatim}

setosa = subset(iris,iris$Species=="setosa")
virgca = subset(iris,iris$Species=="virginica")
versic = subset(iris,iris$Species=="versicolor")
\end{verbatim}
\end{framed}
Next we will construct a basic scatterplot using the entire data set. This will ensure that the correct ranges are used. We will mark each observaton with a dot.
\begin{framed}
\begin{verbatim}
plot(iris$Sepal.Length, iris$Sepal.Width, pch =".")
\end{verbatim}
\end{framed}
Now we will overlay the plot with new points, each specific to each subset.
\begin{framed}
\begin{verbatim}
points(setosa$Sepal.Length, setosa$Sepal.Width,  pch =18, col="red")

points(virgca$Sepal.Length, virgca$Sepal.Width, pch = 17, col="green")

points(versic$Sepal.Length, versic$Sepal.Width, pch = 16, col="blue")
detach(iris)
\end{verbatim}
\end{framed}
\begin{figure}[h!]
\centering
\includegraphics[width=0.85\linewidth]{./irisplot2}
\end{figure}

\newpage
\section{Histograms}
Histograms are used very often in statistics to show the distributions of numeric variables.  Although the basic command for histograms (\texttt{hist()}) in \texttt{R} is simple, getting your histogram to look exactly like you want takes getting to know a few options of the plot. 

\begin{framed}
\begin{verbatim}
hist(iris$Sepal.Length)

\end{verbatim}
\end{framed}
\begin{figure}[h!]
\centering
\includegraphics[width=0.7\linewidth]{./irishist1}
\caption{}
\label{fig:irishist1}
\end{figure}

\subsection{Adding Colour to Histograms}
Colour can be added to a graph using the \textbf{\texttt{``col="}}
\begin{framed}
\begin{verbatim}
hist(iris$Sepal.Length,col="lightblue")
\end{verbatim}
\end{framed}
\begin{center}
\includegraphics[scale=0.65]{histCol}
\end{center}

More than one colour may be specified. We will construct a \textbf{\textit{palette}}.
\begin{itemize}
\item We do not have to attach the \texttt{iris} data file again.
\item A full list of the 657 colours supported by \texttt{R} may be found by using the \texttt{\textbf{colours()}} command. 
\end{itemize}

\begin{framed}
\begin{verbatim}
Palette=c("green","blue","orange","midnightblue")
hist(iris$Petal.Length,col=Palette)
\end{verbatim}
\end{framed}

\begin{center}
\includegraphics[scale=0.65]{histCol2}
\end{center}


\newpage
\section{Adding Text}

Text can be added to graphs using the \textbf{\texttt{text()}}  function. This command places text within the graph. The first argument is the x and y co-ordinates of where the text is to be placed, then the text to be used, specified in quotation marks.
(Use location (x,y)=(-2.5,-3) for this example.)


\begin{framed}
\begin{verbatim}
plot(x,y,  xlim=c(-2.4,6.4),ylim=c(-2,5),pch=18,col="red")
text(x=5,y=3,"Text to be Added")
\end{verbatim}
\end{framed}


%----------------------------------------------------------------------%

%----------------------------------------------------------------------%

\subsection{Title and Subtitles}
It is good practice to have a title attached to your scatterplot. The chosen title (``titlename") of the plot is specified using the argument \texttt\textbf{{main="titlename"}}. Text for a subtitle (here ``subtitle text" ) may also be inserted into the plot, using the argument  \texttt\textbf{{sub="subtitletext"}}. Another way to add a title is using the \texttt{title()} function, as a separate command to plot.


\begin{framed}
\begin{verbatim}
Palette=c("lightblue","blue","navyblue","midnightblue")
hist(iris$Petal.Length,col=Palette,
main="Iris Data Set", sub="Distribution of Sepal Lenghts")
\end{verbatim}
\end{framed}
%---IMAGE ATTACH 3


\begin{figure}
\centering
\includegraphics[width=0.7\linewidth]{./irishist4}
\caption{}
\label{fig:irishist4}
\end{figure}


\subsection{Fonts for Labels and Axes}
There are four ways that you can configure the fonts for labels and axes.

\begin{itemize}
\item[1] Normal (default setting)
\item[2] Bold
\item[3] Italic
\item[4] Bold and Italic
\end{itemize}

To configure the fonts, add \texttt{font.lab=..} and \texttt{font.axis=..} to your \texttt{plot} command, specifying which font type you want (e.g. ..\texttt{font.lab= 2, font.axis=4}..)

\subsection{Adding Labels to the Plot Axes}
Rather than using the variable names as the axes labels, you can specify your own, using the \texttt{ylab=("..")} and \texttt{xlab="...")} arguments to the plot function. An example of this is included in the next section.
\subsection{Adding Text to the Plot Margins}

The margins (i.e. sides) of the plot are numbered in a clockwise order. The bottom side is numbered $1$, the left hand side is $2$, the top side is numbered $3$ and the right hand side is $4$.

\begin{framed}
\begin{verbatim}
plot(x,y,main="Scatterplot Example",   sub="subtitle",    
xlab="X variable ", ylab="y variable ")	
mtext("Enhanced Scatterplot", side=4,col="red ")
\end{verbatim}
\end{framed}





\subsection{Adding Lines to a Scatter Plot}
Previously we have used scatter plots to plot bivariate data. They were constructed using the \texttt{\textbf{plot()}} command.  

Using the \texttt{\textbf{abline()}} command, we can add lines to our scatter plots. We specify the argument according to the type of line required. A demonstration of three types of line is provided below.



\begin{framed}
\begin{verbatim}

plot(mtcars$mpg,mtcars$wt,pch=18,col="red")
title("Cars Data Set",sub="MPG v Weight")

# Now to Add Lines
abline(v=mean(mtcars$mpg))
abline(h=mean(mtcars$wt))
abline(a=5,b=-1/6)
\end{verbatim}
\end{framed}

\begin{center}[h!]
\includegraphics[scale=0.55]{linesDemo}
\end{center}

Additionally we change the colour of the added lines, by specifying a colour in the \texttt{\textbf{col=}} argument. Finally we can change the line type to one of four possible types, using the lty argument.
The line types are follows 
\begin{itemize}
\item \texttt{\textbf{lty =1}}   Normal full line (default)
\item \texttt{\textbf{lty =2}}  Dashed line 
\item \texttt{\textbf{lty =3}}   Dotted line
\item \texttt{\textbf{lty =4}}   Dash-dot line
\end{itemize}


\subsubsection{Adding a Regresion Line to a Plot}
In Regression models, scatterplots are often used to depict the relationship between the predictor variable ($x$) and a response variable ($y$).


\begin{framed}
\begin{verbatim}
Fit1 = lm(mpg~wt,data=mtcars)
coef(Fit1)
#(Intercept)          wt 
#  37.285126   -5.344472 
plot(wt,mpg,pch=18,col="red")
abline(coef(Fit1))
\end{verbatim}
\end{framed}
\begin{center}
\includegraphics[scale=0.35]{coefPlot}
\end{center}

%\subsection{Line Types and Line Width}
%To specify a line width, we use the \texttt\textbf{{lwd=}} argument.

\subsection{Combining Plots}
Multiple plots can be combined into a single graph, arranged vertically, horizontally or in a grid.

This can be implemented with \texttt{\textbf{par( )}} function, specifying the arrangment using the \texttt{\textbf{mfrow=}} argeument.

Suppose we require a histogram above a boxplot (with horizontal orientation). If we consider this as a grid, we have an array of two rows, with one column.

\begin{framed}
\begin{verbatim}
# 2 plots arranged in 2 rows and 1 column


par(mfrow=c(2,1))


hist(mtcars$wt, main="Histogram of wt",col=c("lightblue","blue"))
boxplot(mtcars$
wt, main="Boxplot of wt",horizontal=TRUE,col="lightblue")

\end{verbatim}
\end{framed}

\begin{center}[h1]
\includegraphics[scale=0.65]{combiPlot}
\end{center}
