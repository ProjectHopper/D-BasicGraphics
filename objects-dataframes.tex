\documentclass[11pt]{article} % use larger type; default would be 10pt

\usepackage[utf8]{inputenc} % set input encoding (not needed with XeLaTeX)
\usepackage{geometry} % to change the page dimensions
\geometry{a4paper} % or letterpaper (US) or a5paper or....
\usepackage{framed}
\usepackage{graphicx} % support the \includegraphics command and options
\usepackage{booktabs} % for much better looking tables
\usepackage{array} % for better arrays (eg matrices) in maths
\usepackage{paralist} % very flexible & customisable lists (eg. enumerate/itemize, etc.)
\usepackage{verbatim} % adds environment for commenting out blocks of text & for better verbatim
\usepackage{subfig} % make it possible to include more than one captioned figure/table in a single float
% These packages are all incorporated in the memoir class to one degree or another...

%%% HEADERS & FOOTERS
\usepackage{fancyhdr} % This should be set AFTER setting up the page geometry
\pagestyle{fancy} % options: empty , plain , fancy
\renewcommand{\headrulewidth}{0pt} % customise the layout...
\lhead{MS4024}\chead{R Programming}\rhead{Data Frames}
\lfoot{}\cfoot{\thepage}\rfoot{}

%%% SECTION TITLE APPEARANCE
\usepackage{sectsty}
\allsectionsfont{\sffamily\mdseries\upshape} % (See the fntguide.pdf for font help)
% (This matches ConTeXt defaults)

%%% ToC (table of contents) APPEARANCE
\usepackage[nottoc,notlof,notlot]{tocbibind} % Put the bibliography in the ToC
\usepackage[titles,subfigure]{tocloft} % Alter the style of the Table of Contents
\renewcommand{\cftsecfont}{\rmfamily\mdseries\upshape}
\renewcommand{\cftsecpagefont}{\rmfamily\mdseries\upshape} % No bold!

\begin{document}
\tableofcontents


\section{Data Frames}
Technically, a data frame in \texttt{R} is a very important type of data object;  a type of table where the typical use employs the rows as observations (or cases) and the columns as variables.
Inter alia, a data frame differs from a matrix in that it can contain character values. Many data sets are stored as data frames.

Let us consider the following two variables; \textbf{age} and \textbf{height}.

\begin{framed}
\begin{verbatim}
age=18:29
\end{verbatim}
\end{framed}
\begin{verbatim}
> age
[1] 18 19 20 21 22 23 24 25 26 27 28 29
\end{verbatim}
In similar fashion, we entered the average heights in a vector called height.
\begin{framed}
\begin{verbatim}
height=c(76.1,77,78.1,78.2,78.8,79.7,79.9,81.1,81.2,81.8,82.8,83.5)
\end{verbatim}
\end{framed}
\begin{verbatim}
> height
[1] 76.1 77.0 78.1 78.2 78.8 79.7 79.9 81.1 81.2 81.8 82.8 83.5
\end{verbatim}



%-----------------------------------------------------------------------------%

We will now use \texttt{R}'s \texttt{data.frame()} command to create our first data frame and store the results in the data frame “\textit{village}”.
\begin{framed}
\begin{verbatim}
village=data.frame(age=age,height=height)
\end{verbatim}
\end{framed}

How do we access the data in each column? One way is to state the variable containing the data frame, followed by a dollar sign, then the name of the column we wish to access (as with lists . For example, if we wanted to access the data in the "age" column, we would do the following:
\begin{verbatim}
 > village$age
 [1] 18 19 20 21 22 23 24 25 26 27 28 29
\end{verbatim}
The additional typing required by the "dollar sign" notation can quickly become tiresome, so \texttt{R} provides the ability to "attach" the variables in the data frame to our workspace.

\begin{verbatim}
> attach(village)
\end{verbatim}

Let's re-examine our workspace. ( The \texttt{ls()} command lists all data objects in the workspace )
\begin{verbatim}
> ls()
[1] "village"
\end{verbatim}

No evidence of the variables in the workspace. However, R has made copies of the variables in the columns of the data frame, and most importantly, we can access them without the "dollar notation."
\begin{verbatim}
> age
 [1] 18 19 20 21 22 23 24 25 26 27 28 29
> height
 [1] 76.1 77.0 78.1 78.2 78.8 79.7 79.9 81.1 81.2 81.8 82.8
[12] 83.5
\end{verbatim}
%-------------------------------------------------------------------%
\subsection{using rownames and colnames}
Previously we have seen \texttt{rownames()} and \texttt{colnames()} to determine the names from an existing data frame. We can use these commands to create names for a new data frame.
Vectors can be bound together either by row or by column.
\begin{verbatim}
> X=1:3; Y=4:6
> cbind(X,Y)
     X Y
[1,] 1 4
[2,] 2 5
[3,] 3 6
>
> rbind(X,Y)
  [,1] [,2] [,3]
X    1    2    3
Y    4    5    6
\end{verbatim}

We can use this approach to create a data frame :

\begin{verbatim}
> data.frame(rbind(X,Y))
  X1 X2 X3
X  1  2  3
Y  4  5  6
\end{verbatim}

We can then use rownames() and colnames() to assign meaningful names to this data frame.
\end{document}