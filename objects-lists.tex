
%================================================================================= %
\begin{frame}[fragile]
	\frametitle{ Lists }
	Many data objects returned as output are structured as lists. 
	(Recall the output from the eigen() function.)
	An  R list is an object consisting of an ordered collection of objects known as its components. 
	There is no particular need for the components to be of the same mode or type, and, for example, a list could consist of a numeric vector, a logical value, a matrix, a complex vector, a character array, a function, and so on.
	
\end{frame}
%================================================================================= %
\begin{frame}[fragile]
	\frametitle{ Lists }
	Here is a simple example of how to make a list: 
	\begin{verbatim}
	> myList <- list(name="Oscar", Boss="LouLou", no.children=3,                   child.ages=c(4,7,9))
	
	\end{verbatim}     
	Components are always numbered and may always be referred to as such. 
	Thus if myList is the name of a list with 4 components, these may be individually referred to as myList[[1]], myList[[2]], myList[[3]] and myList[[4]]. 
\end{frame}
%================================================================================= %
\begin{frame}[fragile]
	If myList[[4]] is a vector, then myList[[4]][1] is its first entry. 
	
	> myList
	$name
	[1] "Oscar"
	
	$Boss
	[1] "LouLou"
	
	$no.children
	[1] 3
	
	$child.ages
	[1] 4 7 9
\end{frame}
%================================================================================= %
\begin{frame}[fragile]
	> myList[[1]]
	[1] "Oscar"
	> myList[[4]][1]
	[1] 4
\end{frame}
%================================================================================= %
\begin{frame}[fragile]
	
	The function length(myList) gives the number of (top level) components that the list has. 
	
	Components of lists may also be named, and in this case the component may be referred to either by giving the component name as a character string in place of the number in double square brackets, or, more conveniently, by giving an expression of the form 
	
	\begin{verbatim}
	> name$component_name
	\end{verbatim}
	
	for the same thing. 
\end{frame}
%================================================================================= %
\begin{frame}[fragile]
	This is a very useful convention as it makes it easier to get the right component if you forget the number. So in the simple example given above: 
	\begin{itemize}
		\item	myList\$name is the same as myList[[1]] and is the string "Oscar", 
		\item	myList\$Boss is the same as myList[[2]] and is the string "LouLou", 
		\item	myList\$child.ages[1] is the same as myList[[4]][1] and is the number 4.
	\end{itemize}
\end{frame}
%================================================================================= %
\begin{frame}[fragile]
	This dollar sign operator is very useful , particularly when looking at the output of a complex statistical function.
	To find out the names assigned to a list use the command names().
	\begin{framed}
		\begin{verbatim}
		> names(myList)
		[1] "name"        "Boss"        "no.children" "child.ages" 
		> myList$name
		[1] "Oscar"
		\end{verbatim}
	\end{framed} 
	
\end{frame}
%================================================================================= %
\end{document}