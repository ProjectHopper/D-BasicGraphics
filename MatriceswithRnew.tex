\documentclass[a4paper,12pt]{article}
%%%%%%%%%%%%%%%%%%%%%%%%%%%%%%%%%%%%%%%%%%%%%%%%%%%%%%%%%%%%%%%%%%%%%%%%%%%%%%%%%%%%%%%%%%%%%%%%%%%%%%%%%%%%%%%%%%%%%%%%%%%%%%%%%%%%%%%%%%%%%%%%%%%%%%%%%%%%%%%%%%%%%%%%%%%%%%%%%%%%%%%%%%%%%%%%%%%%%%%%%%%%%%%%%%%%%%%%%%%%%%%%%%%%%%%%%%%%%%%%%%%%%%%%%%%%
\usepackage{eurosym}
\usepackage{vmargin}
\usepackage{amsmath}
\usepackage{graphics}
\usepackage{epsfig}
\usepackage{subfigure}
\usepackage{fancyhdr}
%\usepackage{listings}
\usepackage{framed}
\usepackage{graphicx}

\setcounter{MaxMatrixCols}{10}
%TCIDATA{OutputFilter=LATEX.DLL}
%TCIDATA{Version=5.00.0.2570}
%TCIDATA{<META NAME="SaveForMode" CONTENT="1">}
%TCIDATA{LastRevised=Wednesday, February 23, 2011 13:24:34}
%TCIDATA{<META NAME="GraphicsSave" CONTENT="32">}
%TCIDATA{Language=American English}

\pagestyle{fancy}
\setmarginsrb{20mm}{0mm}{20mm}{25mm}{12mm}{11mm}{0mm}{11mm}
\lhead{Rstats} \rhead{Kevin O'Brien}
\chead{Introduction to R Programming}
%\input{tcilatex}

\begin{document}
\section{Matrices}
A matrix refers to a numeric array of rows and columns.

One of the easiest ways to create a matrix is to combine vectors of equal
length using \texttt{cbind()}, meaning "column bind". Alternatively one can use  \texttt{rbind()}, meaning ``row bind".


\subsubsection{Matrices Inversion}
\subsubsection{Matrices Multiplication}

%======================================================================================================= %	

\section{Matrices}
\subsection{Creating a matrix}
\begin{itemize}
\item Matrices can be created using the \texttt{matrix()} command. 
\item The arguments to be supplied are 
\begin{enumerate}
\item vector of values to be entered
\item Dimensions of the matrix, specifying either the numbers of rows or columns.
\end{enumerate}
\item 
Additionally you can specify if the values are to be allocated by row or column. By default they are allocated \textbf{by column}.
\begin{verbatim}
x=c(1,4,5,6,4,5,5,7,9)		# 9 elements
A=matrix(x,nrow=3)		#3 by 3 matrix. Values assigned by column.
A

B= matrix(c(1,6,7,0.6,0.5,0.3,1,2,1),ncol=3,byrow =TRUE)
B				          #3 by 3 matrix. Values assigned by row.
\end{verbatim}	

%============================================================== % 
\subsection*{Transpose Matrix}
If you have assigned values by column, but require that they are assigned by row, you can use the transpose function
\begin{framed}
\begin{verbatim}
t().
t(A)				# Transpose
A=t(A)	
\end{verbatim}
\end{framed}
%============================================================== %
Another methods of creating a matrix is to "bind" a number of vectors together, either by row or by column. The commands are rbind() and cbind() respectively.
\begin{verbatim}
x1 =c(1,2) ; x2 = c(3,6)
rbind(x1,x2)
cbind(x1,x2)
\end{verbatim}

%============================================================== %
\newpage
\subsection*{Access Row or Columns}

Particular rows and columns can be accessed by specifying the row number or column number, leaving the other value blank.
\begin{framed}
\begin{verbatim}
A[1,]	  # access first row of A
B[,2]   # access first column of B
\end{verbatim}
\end{framed}
%============================================================== %
\newpage
\subsection{Addition and subtractions}
For matrices, addition and subtraction works on an element-wise basis. The first elements of the respective matrices are added, and so on.
\begin{framed}
\begin{verbatim}
A+B
A-B
\end{verbatim}
\end{framed}

\subsection{Matrix Multiplication}
To multiply matrices, we require a special operator for matrices; $"\%*\%"$.
If we just used the normal multiplication, we would get an element-wise multiplication.
\begin{framed}
\begin{verbatim}
A * B  		#Element-wise multiplication
A %*% B  	#Matrix multiplication
\end{verbatim}
\end{framed}

We can compute crossproducts using the \texttt{crossprod()} command. If only one matrix is used it
\begin{verbatim}
crossprod(A,B) 		# A'B
crossprod(A) 			# A'A
\end{verbatim}
%============================================================= %
\newpage
\subsection{Diagonals}
The \texttt{diag()} command is a very versatile function for using matrices.
It can be used to create a diagonal matrix with elements of a vector in the principal diagonal. For an existing matrix, it can be used to return a vector containing the elements of the principal diagonal.


Most importantly, if k is a scalar, \texttt{diag()} will create a $k \times k$identity matrix.
\begin{framed}
\begin{verbatim}
Vec2=c(1,2,3)
diag(Vec2)	#	Constructs a diagonal matrix based on values of Vec2
diag(A)	#        Returns diagonal elements of A as a vector
diag(3)	#	creates a 3 x 3 identity matrix
diag(diag(A)) #  	Diagonal matrix D of matrix A ( Jacobi Method)
\end{verbatim}
\end{framed}


\newpage

\subsection{Determinants, Inverse Matrices and solving Linear systems}
To compute the determinant of a square matrix, we simply use the det() command
det(A)
det(B)
To find the inverse of a square matrix, we use the solve() command, specifying only the matrix in question
solve(A)

To solve a system of linear equations in the form Ax=b , where A is a square matrix, and b is a column vector of known values, we use the solve() command to determine the values of the unknown vector x.
\begin{verbatim}
b=vec2  # from before
solve(A, b)
\end{verbatim}
Row and Column Statistics.
Statistic on the rows and columns can easily be computed if required.
\begin{verbatim}
rowMeans(A)  # Returns vector of row means.
rowSums(A)  # Returns vector of row sums.
colMeans(A)  # Returns vector of column means.
colSums(A)  # Returns vector of coumn means.
\end{verbatim}
\newpage
\section{Eigenvalues and Eigenvectors}
The eigenvalues and eigenvectors can be computed using the eigen() function.  A data object is created.
This is a very important type of matrix analysis, and many will encounter it again in future modules.
\begin{verbatim}
Y = eigen(A)
names(Y)
"	y$val are the eigenvalues of A
"	y$vec are the eigenvectors of A
\end{verbatim}



\end{document}